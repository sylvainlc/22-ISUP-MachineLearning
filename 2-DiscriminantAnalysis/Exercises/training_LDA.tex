\documentclass[a4paper,10pt,fleqn]{article}

\usepackage{a4wide,amsmath,amsthm,amssymb,bbm,fancyhdr}
\usepackage{ifthen,color,enumerate,comment,dsfont,pdfsync,framed,todonotes,enumitem}

\newcommand{\titre}[1]{\textbf{\textsc{#1}}}

\RequirePackage[T1]{fontenc}

\usepackage[latin1]{inputenc}
\usepackage{graphicx}
\usepackage{dsfont}
\usepackage{enumitem}
\newcommand{\eqsp}{\,}
\newcommand{\R}{\ensuremath{\mathbb{R}}}
\newcommand{\calF}{\mathcal{F}}
\newcommand{\rmd}{\mathrm{d}}
\newcommand{\N}{\mathbb{N}}
\newcommand{\rset}{\ensuremath{\mathbb{R}}}
\renewcommand{\P}{\ensuremath{\operatorname{P}}}
\newcommand{\bP}{\mathbb{P}}
\newcommand{\E}{\ensuremath{\mathbb{E}}}
\newcommand{\rme}{\ensuremath{\mathrm{e}}}
\newcommand{\calH}{\ensuremath{\mathcal{H}}}
\newcommand{\xset}{\ensuremath{\mathsf{X}}}
\newcommand{\V}{\ensuremath{\mathbb{V}}}
\newcommand{\Sb}{\ensuremath{\mathbb{S}}}
\newcommand{\gaus}{\ensuremath{\mathcal{N}}}
\newcommand{\HH}{\ensuremath{\mathcal{H}}}
\newcommand{\F}{\ensuremath{\mathcal{F}}}
\newcommand{\W}{\ensuremath{\mathcal{W}}}
\newcommand{\X}{\ensuremath{\mathcal{X}}}
\newcommand{\1}{\ensuremath{\mathbbm{1}}}
\newcommand{\dlim}{\ensuremath{\stackrel{\mathcal{L}}{\longrightarrow}}}
\newcommand{\plim}{\ensuremath{\stackrel{\mathrm{P}}{\longrightarrow}}}
\newcommand{\PP}{\ensuremath{\mathbb{P}}}
\newcommand{\p}{\ensuremath{\mathbb{P}}}
\newcommand{\eps}{\varepsilon}
\newcommand{\bE}{\mathbb{E}}
\newcommand{\pa}[1]{\left(#1\right)}
\newcommand{\hatk}{\widehat K}
\newcommand{\f}{\varphi}
\newcommand{\Id}{\textsf{Id}}
\newcommand{\bfU}{\mathbf{U}}
\newcommand{\bfX}{\mathbf{X}}
\newcommand{\bfs}{\mathbf{\Sigma}}
\newcommand{\bfA}{\mathbf{A}}
\newcommand{\bfV}{\mathbf{V}}
\newcommand{\bfB}{\mathbf{B}}
\newcommand{\bfI}{\mathbf{I}}
\newcommand{\bfD}{\mathbf{D}}
\newcommand{\bfK}{\mathbf{K}}
\newcommand{\argmin}{\mathop{\textrm{argmin}}}
\newcommand{\argmax}{\mathop{\textrm{argmax}}}
\newcommand{\crit}{\mathop{\textrm{crit}}}
\newcommand{\C}{\mathcal{C}}
\newcommand{\pc}{\pi_{\mathcal{C}}}


% Style
%\pagestyle{fancyplain}
\renewcommand{\sectionmark}[1]{\markright{#1}}
\renewcommand{\subsectionmark}[1]{}
%\lhead[\fancyplain{}{\thepage}]{\fancyplain{}{\footnotesize {\sf
%MAP569 Machine Learning II, \thisyear %/ \rightmark
%}}}
%\rhead[\fancyplain{}{\footnotesize {\sf MAT4506 Introduction to machine learning, \thisyear %/ \rightmark
%}}]{\fancyplain{}{\thepage}}


\begin{document}

\noindent Machine learning \hfill ISUP - Sorbonne Universit\'e \\
 2022-2023

\noindent\hrulefill

\begin{center}
\textsc{Discriminant analysis}
\end{center}
\hrulefill

\section{Classification error}

Linear discriminant analysis assumes that the random variables $(X,Y)\in \rset^d\times\{0,1\}$ have the following distribution. For all $A\in \mathcal{B}(\mathbb{R}^d)$ and all $y\in\{0,1\}$,
$$
\bP\left(X\in A;Y=y\right) = \pi_y \int_A g_y(x) \rmd x\eqsp,
$$
where $\pi_{0}$ and $\pi_1$ are positive real numbers such that $\pi_0+\pi_{1}=1$ and $g_0$ (resp. $g_1$) is the probability density of a Gaussian random variable with mean $\mu_0\in\rset^d$ (resp. $\mu_1)$ and positive definite covariance matrix $\Sigma_0\in \rset^{d\times d}$ (resp. $\Sigma_1$).  Define the classifier $h_{*}:\mathbb{R}^d\to\{0,1\}$ by
$$
h_{*}:x \mapsto \1_{\{\pi_{1}g_{1}(x)>\pi_{0}g_{0}(x)\}}\eqsp.
$$
\begin{enumerate}
\item Give the distribution of the random variable $X$ and prove that 
$$
\bP(h_{*}(X)\neq Y)=\min_{h:\rset^d\to \{0,1\}}\left\{\bP(h(X)\neq Y)\right\}\eqsp.
$$
\item Assume that  $\mu_0\ne\mu_1$. Prove that when $\Sigma_0=\Sigma_1=\Sigma$, for all $x\in\rset^d$,
$$
h_{*}(x) = 1 \Leftrightarrow (\mu_{1}-\mu_{0})^\top\Sigma^{-1}\left(x-{\mu_{1}+\mu_{0}\over2}\right)>\log(\pi_{0}/\pi_{1})\eqsp.
$$
Provide a geometrical interpretation.
\item Prove that when $\pi_{1}=\pi_{0}$, 
$$
\bP(h_{*}(X)=1|Y=0)=\Phi(-d(\mu_{1},\mu_{0})/2)\eqsp,
$$
where $\Phi$ is the cumulative distribution function of a standard Gaussian random variable  and 
$$
d(\mu_{1},\mu_{0})^2=(\mu_{1}-\mu_{0})^\top\Sigma^{-1}(\mu_{1}-\mu_{0})\eqsp.
$$
\item Assume now that $\Sigma_{1}\neq \Sigma_{0}$. What is the nature of the frontier between $\{x\eqsp;\eqsp h_{*}(x)=1\}$ and $\{x\eqsp;\eqsp h_{*}(x)=0\}$?
\end{enumerate}

\section{Maximum likelihood estimation}
We assume that the joint distribution of $(X,Y)$ belongs to a family of distributions parametrized by a vector $\theta$ with real components. For $k\in\{-1,1\}$, write $\pi_k = \bP(Y = k)$. Assume that conditionally on the event $\{Y = k\}$, $X$ has a Gaussian distribution with mean $\mu_k \in\rset^d$ and covariance matrix $\Sigma\in \rset^{d\times d}$, whose density is denoted $g_k$. In this case, the parameter $\theta=(\pi_1, \mu_1,\mu_{-1}, \Sigma)$ belongs to the set $\Theta= [0,1] \times \rset^d \times \rset^d \times \rset^{d \times d}$. The parameter $\pi_{-1}$ is not part of the components of $\theta$ since $\pi_{-1}=1-\pi_{1}$. In this case, the parameter $\theta=(\pi_1, \mu_1,\mu_{-1}, \Sigma)$ belongs to the set $\Theta= [0,1] \times \mathbb{R}^d \times \mathbb{R}^d \times \mathbb{R}^{d \times d}$. The parameter $\pi_{-1}$ is not part of the components of $\theta$ since $\pi_{-1}=1-\pi_{1}$. 

When $\Sigma$ and $\mu_1$ and $\mu_{-1}$ are unknown, the  discriminant analysis classifier cannot be computed explicitely. Assume that  $(X_i,Y_i)_{1\leqslant i\leqslant n}$ are independent observations with the same distribution as $(X,Y)$.
\begin{enumerate}
\item Write the joint loglikelihood of the observations.
\item Let $M_d$ be the space of real-valued $d \times d$ symmetric positive matrices. Show that the function $\Sigma \mapsto \log \det \Sigma$ is concave on $M_d$.
\item Show that the derivative of the real valued function $\Sigma \mapsto \log\mathrm{det}(\Sigma)$ defined on $\rset^{d\times d}$ is given by:
$$
\partial_{\Sigma}\{\log\mathrm{det}(\Sigma)\}= \Sigma^{-1}\eqsp,
$$
where, for all real valued function $f$ defined on $\rset^{d\times d}$, $\partial_{\Sigma}f(\Sigma)$ denotes the $\rset^{d\times d}$ matrix such that for all $1\leqslant i,j\leqslant d$, $\{\partial_{\Sigma}f(\Sigma)\}_{i,j}$ is the partial derivative of $f$ with respect to $\Sigma_{i,j}$.
\item Provide the maximum likehood estimator of $\theta$.
\item How do you suggest to use this estimator to build a classifier ?
\end{enumerate}

\end{document}